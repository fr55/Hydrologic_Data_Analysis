\documentclass[]{article}
\usepackage{lmodern}
\usepackage{amssymb,amsmath}
\usepackage{ifxetex,ifluatex}
\usepackage{fixltx2e} % provides \textsubscript
\ifnum 0\ifxetex 1\fi\ifluatex 1\fi=0 % if pdftex
  \usepackage[T1]{fontenc}
  \usepackage[utf8]{inputenc}
\else % if luatex or xelatex
  \ifxetex
    \usepackage{mathspec}
  \else
    \usepackage{fontspec}
  \fi
  \defaultfontfeatures{Ligatures=TeX,Scale=MatchLowercase}
\fi
% use upquote if available, for straight quotes in verbatim environments
\IfFileExists{upquote.sty}{\usepackage{upquote}}{}
% use microtype if available
\IfFileExists{microtype.sty}{%
\usepackage{microtype}
\UseMicrotypeSet[protrusion]{basicmath} % disable protrusion for tt fonts
}{}
\usepackage[margin=2.54cm]{geometry}
\usepackage{hyperref}
\hypersetup{unicode=true,
            pdftitle={Assignment 8: Mapping},
            pdfauthor={Felipe Raby Amadori},
            pdfborder={0 0 0},
            breaklinks=true}
\urlstyle{same}  % don't use monospace font for urls
\usepackage{color}
\usepackage{fancyvrb}
\newcommand{\VerbBar}{|}
\newcommand{\VERB}{\Verb[commandchars=\\\{\}]}
\DefineVerbatimEnvironment{Highlighting}{Verbatim}{commandchars=\\\{\}}
% Add ',fontsize=\small' for more characters per line
\usepackage{framed}
\definecolor{shadecolor}{RGB}{248,248,248}
\newenvironment{Shaded}{\begin{snugshade}}{\end{snugshade}}
\newcommand{\AlertTok}[1]{\textcolor[rgb]{0.94,0.16,0.16}{#1}}
\newcommand{\AnnotationTok}[1]{\textcolor[rgb]{0.56,0.35,0.01}{\textbf{\textit{#1}}}}
\newcommand{\AttributeTok}[1]{\textcolor[rgb]{0.77,0.63,0.00}{#1}}
\newcommand{\BaseNTok}[1]{\textcolor[rgb]{0.00,0.00,0.81}{#1}}
\newcommand{\BuiltInTok}[1]{#1}
\newcommand{\CharTok}[1]{\textcolor[rgb]{0.31,0.60,0.02}{#1}}
\newcommand{\CommentTok}[1]{\textcolor[rgb]{0.56,0.35,0.01}{\textit{#1}}}
\newcommand{\CommentVarTok}[1]{\textcolor[rgb]{0.56,0.35,0.01}{\textbf{\textit{#1}}}}
\newcommand{\ConstantTok}[1]{\textcolor[rgb]{0.00,0.00,0.00}{#1}}
\newcommand{\ControlFlowTok}[1]{\textcolor[rgb]{0.13,0.29,0.53}{\textbf{#1}}}
\newcommand{\DataTypeTok}[1]{\textcolor[rgb]{0.13,0.29,0.53}{#1}}
\newcommand{\DecValTok}[1]{\textcolor[rgb]{0.00,0.00,0.81}{#1}}
\newcommand{\DocumentationTok}[1]{\textcolor[rgb]{0.56,0.35,0.01}{\textbf{\textit{#1}}}}
\newcommand{\ErrorTok}[1]{\textcolor[rgb]{0.64,0.00,0.00}{\textbf{#1}}}
\newcommand{\ExtensionTok}[1]{#1}
\newcommand{\FloatTok}[1]{\textcolor[rgb]{0.00,0.00,0.81}{#1}}
\newcommand{\FunctionTok}[1]{\textcolor[rgb]{0.00,0.00,0.00}{#1}}
\newcommand{\ImportTok}[1]{#1}
\newcommand{\InformationTok}[1]{\textcolor[rgb]{0.56,0.35,0.01}{\textbf{\textit{#1}}}}
\newcommand{\KeywordTok}[1]{\textcolor[rgb]{0.13,0.29,0.53}{\textbf{#1}}}
\newcommand{\NormalTok}[1]{#1}
\newcommand{\OperatorTok}[1]{\textcolor[rgb]{0.81,0.36,0.00}{\textbf{#1}}}
\newcommand{\OtherTok}[1]{\textcolor[rgb]{0.56,0.35,0.01}{#1}}
\newcommand{\PreprocessorTok}[1]{\textcolor[rgb]{0.56,0.35,0.01}{\textit{#1}}}
\newcommand{\RegionMarkerTok}[1]{#1}
\newcommand{\SpecialCharTok}[1]{\textcolor[rgb]{0.00,0.00,0.00}{#1}}
\newcommand{\SpecialStringTok}[1]{\textcolor[rgb]{0.31,0.60,0.02}{#1}}
\newcommand{\StringTok}[1]{\textcolor[rgb]{0.31,0.60,0.02}{#1}}
\newcommand{\VariableTok}[1]{\textcolor[rgb]{0.00,0.00,0.00}{#1}}
\newcommand{\VerbatimStringTok}[1]{\textcolor[rgb]{0.31,0.60,0.02}{#1}}
\newcommand{\WarningTok}[1]{\textcolor[rgb]{0.56,0.35,0.01}{\textbf{\textit{#1}}}}
\usepackage{graphicx,grffile}
\makeatletter
\def\maxwidth{\ifdim\Gin@nat@width>\linewidth\linewidth\else\Gin@nat@width\fi}
\def\maxheight{\ifdim\Gin@nat@height>\textheight\textheight\else\Gin@nat@height\fi}
\makeatother
% Scale images if necessary, so that they will not overflow the page
% margins by default, and it is still possible to overwrite the defaults
% using explicit options in \includegraphics[width, height, ...]{}
\setkeys{Gin}{width=\maxwidth,height=\maxheight,keepaspectratio}
\IfFileExists{parskip.sty}{%
\usepackage{parskip}
}{% else
\setlength{\parindent}{0pt}
\setlength{\parskip}{6pt plus 2pt minus 1pt}
}
\setlength{\emergencystretch}{3em}  % prevent overfull lines
\providecommand{\tightlist}{%
  \setlength{\itemsep}{0pt}\setlength{\parskip}{0pt}}
\setcounter{secnumdepth}{0}
% Redefines (sub)paragraphs to behave more like sections
\ifx\paragraph\undefined\else
\let\oldparagraph\paragraph
\renewcommand{\paragraph}[1]{\oldparagraph{#1}\mbox{}}
\fi
\ifx\subparagraph\undefined\else
\let\oldsubparagraph\subparagraph
\renewcommand{\subparagraph}[1]{\oldsubparagraph{#1}\mbox{}}
\fi

%%% Use protect on footnotes to avoid problems with footnotes in titles
\let\rmarkdownfootnote\footnote%
\def\footnote{\protect\rmarkdownfootnote}

%%% Change title format to be more compact
\usepackage{titling}

% Create subtitle command for use in maketitle
\providecommand{\subtitle}[1]{
  \posttitle{
    \begin{center}\large#1\end{center}
    }
}

\setlength{\droptitle}{-2em}

  \title{Assignment 8: Mapping}
    \pretitle{\vspace{\droptitle}\centering\huge}
  \posttitle{\par}
    \author{Felipe Raby Amadori}
    \preauthor{\centering\large\emph}
  \postauthor{\par}
    \date{}
    \predate{}\postdate{}
  

\begin{document}
\maketitle

\hypertarget{overview}{%
\subsection{OVERVIEW}\label{overview}}

This exercise accompanies the lessons in Hydrologic Data Analysis on
mapping

\hypertarget{directions}{%
\subsection{Directions}\label{directions}}

\begin{enumerate}
\def\labelenumi{\arabic{enumi}.}
\tightlist
\item
  Change ``Student Name'' on line 3 (above) with your name.
\item
  Work through the steps, \textbf{creating code and output} that fulfill
  each instruction.
\item
  Be sure to \textbf{answer the questions} in this assignment document.
\item
  When you have completed the assignment, \textbf{Knit} the text and
  code into a single pdf file.
\item
  After Knitting, submit the completed exercise (pdf file) to the
  dropbox in Sakai. Add your last name into the file name (e.g.,
  ``A08\_Salk.html'') prior to submission.
\end{enumerate}

The completed exercise is due on 23 October 2019 at 9:00 am.

\hypertarget{setup}{%
\subsection{Setup}\label{setup}}

\begin{enumerate}
\def\labelenumi{\arabic{enumi}.}
\tightlist
\item
  Verify your working directory is set to the R project file,
\item
  Load the tidyverse, lubridate, cowplot, LAGOSNE, sf, maps, and viridis
  packages.
\item
  Set your ggplot theme (can be theme\_classic or something else)
\item
  Load the lagos database, the USA rivers water features shape file, and
  the HUC6 watershed shape file.
\end{enumerate}

\begin{Shaded}
\begin{Highlighting}[]
\CommentTok{# 1. Verify your working directory is set to the R project file}
\KeywordTok{getwd}\NormalTok{()}
\end{Highlighting}
\end{Shaded}

\begin{verbatim}
## [1] "C:/Users/Felipe/OneDrive - Duke University/1. DUKE/Ramos 3 Semestre/Hydrologic_Data_Analysis"
\end{verbatim}

\begin{Shaded}
\begin{Highlighting}[]
\CommentTok{# 2. Load the tidyverse, lubridate, cowplot, LAGOSNE, sf, maps, and viridis packages }

\NormalTok{packages <-}\StringTok{ }\KeywordTok{c}\NormalTok{(}
  \StringTok{"ggplot2"}\NormalTok{, }
  \StringTok{"tidyverse"}\NormalTok{, }
  \StringTok{"lubridate"}\NormalTok{, }
  \StringTok{"cowplot"}\NormalTok{,}
  \StringTok{"LAGOSNE"}\NormalTok{,}
  \StringTok{"sf"}\NormalTok{,}
  \StringTok{"maps"}\NormalTok{,}
  \StringTok{"viridis"}
\NormalTok{  )}
\KeywordTok{invisible}\NormalTok{(}
  \KeywordTok{suppressPackageStartupMessages}\NormalTok{(}
    \KeywordTok{lapply}\NormalTok{(packages, library, }\DataTypeTok{character.only =} \OtherTok{TRUE}\NormalTok{)}
\NormalTok{    )}
\NormalTok{  ) }

\CommentTok{# 3. Set your ggplot theme (can be theme_classic or something else)}
\NormalTok{felipe_theme <-}\StringTok{ }\KeywordTok{theme_light}\NormalTok{(}\DataTypeTok{base_size =} \DecValTok{12}\NormalTok{) }\OperatorTok{+}
\StringTok{  }\KeywordTok{theme}\NormalTok{(}\DataTypeTok{axis.text =} \KeywordTok{element_text}\NormalTok{(}\DataTypeTok{color =} \StringTok{"grey8"}\NormalTok{), }
        \DataTypeTok{legend.position =} \StringTok{"right"}\NormalTok{, }\DataTypeTok{plot.title =} \KeywordTok{element_text}\NormalTok{(}\DataTypeTok{hjust =} \FloatTok{0.5}\NormalTok{)) }
\KeywordTok{theme_set}\NormalTok{(felipe_theme)}

\CommentTok{# 4. Load the lagos database, the USA rivers water features shape file, and the HUC6}
\CommentTok{# watershed shape file.}

\NormalTok{LAGOSdata <-}\StringTok{ }\KeywordTok{lagosne_load}\NormalTok{()}
\end{Highlighting}
\end{Shaded}

\begin{verbatim}
## Warning in `_f`(version = version, fpath = fpath): LAGOSNE version
## unspecified, loading version: 1.087.3
\end{verbatim}

\begin{Shaded}
\begin{Highlighting}[]
\NormalTok{HUC6.WS <-}\StringTok{ }\KeywordTok{st_read}\NormalTok{(}\StringTok{"./Data/Raw/Watersheds_Spatial/WBDHU6.dbf"}\NormalTok{)}
\end{Highlighting}
\end{Shaded}

\begin{verbatim}
## Reading layer `WBDHU6' from data source `C:\Users\Felipe\OneDrive - Duke University\1. DUKE\Ramos 3 Semestre\Hydrologic_Data_Analysis\Data\Raw\Watersheds_Spatial\WBDHU6.dbf' using driver `ESRI Shapefile'
## Simple feature collection with 33 features and 15 fields
## geometry type:  MULTIPOLYGON
## dimension:      XY
## bbox:           xmin: -90.6235 ymin: 24.39533 xmax: -75.3981 ymax: 37.52103
## epsg (SRID):    4269
## proj4string:    +proj=longlat +datum=NAD83 +no_defs
\end{verbatim}

\begin{Shaded}
\begin{Highlighting}[]
\NormalTok{USAwaterfeatures <-}\StringTok{ }\KeywordTok{st_read}\NormalTok{(}\StringTok{"./Data/Raw/hydrogl020.dbf"}\NormalTok{)}
\end{Highlighting}
\end{Shaded}

\begin{verbatim}
## Reading layer `hydrogl020' from data source `C:\Users\Felipe\OneDrive - Duke University\1. DUKE\Ramos 3 Semestre\Hydrologic_Data_Analysis\Data\Raw\hydrogl020.dbf' using driver `ESRI Shapefile'
## Simple feature collection with 76975 features and 12 fields
## geometry type:  LINESTRING
## dimension:      XY
## bbox:           xmin: -179.9982 ymin: 17.67469 xmax: 179.9831 ymax: 71.39819
## epsg (SRID):    NA
## proj4string:    NA
\end{verbatim}

\hypertarget{mapping-water-quality-in-lakes}{%
\subsection{Mapping water quality in
lakes}\label{mapping-water-quality-in-lakes}}

Complete the in-class exercise from lesson 15, to map average secchi
depth measurements across states in Maine, considering lake area and
lake depth as predictors for water clarity. Steps here are identical to
the lesson, with the following edits:

\begin{itemize}
\tightlist
\item
  Make sure all your wrangling is done in this document (this includes
  basic wrangling of the LAGOS database)
\item
  In your cowplot, do not adjust the legend items (even though they look
  ugly). Rather, reflect on how you would improve them with additional
  coding.
\item
  For item 9, \textbf{do} run a regression on secchi depth by lake area
  and a separate regression on secchi depth by lake depth. Make
  scatterplots of these relationships. Note that log-transforming one of
  these items may be necessary.
\end{itemize}

\begin{enumerate}
\def\labelenumi{\arabic{enumi}.}
\setcounter{enumi}{4}
\tightlist
\item
  Filter the states and secchi depth datasets so that they contain Maine
  only. For the secchi depth dataset, create a summary dataset with just
  the mean secchi depth.
\end{enumerate}

\begin{Shaded}
\begin{Highlighting}[]
\CommentTok{# generate a map of U.S. states}
\NormalTok{states <-}\StringTok{ }\KeywordTok{st_as_sf}\NormalTok{(}\KeywordTok{map}\NormalTok{(}\DataTypeTok{database =} \StringTok{"state"}\NormalTok{, }\DataTypeTok{plot =} \OtherTok{FALSE}\NormalTok{, }\DataTypeTok{fill =} \OtherTok{TRUE}\NormalTok{, }\DataTypeTok{col =} \StringTok{"white"}\NormalTok{))}

\CommentTok{# filter only states that are included in the LAGOSNE database}
\NormalTok{states.subset <-}\StringTok{ }\KeywordTok{filter}\NormalTok{(states, ID }\OperatorTok\StringTok{ }
\StringTok{                          }\KeywordTok{c}\NormalTok{(}\StringTok{"maine"}\NormalTok{))}

\CommentTok{# load LAGOSNE data frames}
\NormalTok{LAGOSlocus <-}\StringTok{ }\NormalTok{LAGOSdata}\OperatorTok{$}\NormalTok{locus}
\NormalTok{LAGOSstate <-}\StringTok{ }\NormalTok{LAGOSdata}\OperatorTok{$}\NormalTok{state}
\NormalTok{LAGOSnutrient <-}\StringTok{ }\NormalTok{LAGOSdata}\OperatorTok{$}\NormalTok{epi_nutr}
\NormalTok{LAGOSlimno <-}\StringTok{ }\NormalTok{LAGOSdata}\OperatorTok{$}\NormalTok{lakes_limno}

\CommentTok{# Create a data frame to visualize secchi depth}
\NormalTok{LAGOScombined <-}\StringTok{ }
\StringTok{  }\KeywordTok{left_join}\NormalTok{(LAGOSnutrient, LAGOSlocus) }\OperatorTok\StringTok{ }\CommentTok{#locus gives you location}
\StringTok{  }\KeywordTok{left_join}\NormalTok{(., LAGOSlimno) }\OperatorTok
\StringTok{  }\KeywordTok{left_join}\NormalTok{(., LAGOSstate) }\OperatorTok
\StringTok{  }\KeywordTok{filter}\NormalTok{(}\OperatorTok{!}\KeywordTok{is.na}\NormalTok{(state)) }\OperatorTok
\StringTok{  }\KeywordTok{select}\NormalTok{(lagoslakeid, sampledate, secchi, lake_area_ha, maxdepth, nhd_lat, nhd_long, state)}
\end{Highlighting}
\end{Shaded}

\begin{verbatim}
## Joining, by = "lagoslakeid"
\end{verbatim}

\begin{verbatim}
## Joining, by = c("lagoslakeid", "nhdid", "nhd_lat", "nhd_long")
\end{verbatim}

\begin{verbatim}
## Joining, by = "state_zoneid"
\end{verbatim}

\begin{Shaded}
\begin{Highlighting}[]
\NormalTok{LAGOScombined.maine <-}\StringTok{ }\KeywordTok{filter}\NormalTok{(LAGOScombined, state }\OperatorTok{==}\StringTok{ "ME"}\NormalTok{)}

\CommentTok{# create a summary dataset with just the mean secchi depth}
\NormalTok{secchi.summary <-}\StringTok{ }\NormalTok{LAGOScombined.maine }\OperatorTok
\StringTok{  }\KeywordTok{group_by}\NormalTok{(lagoslakeid) }\OperatorTok
\StringTok{  }\KeywordTok{summarise}\NormalTok{(}\DataTypeTok{secchi.mean =} \KeywordTok{mean}\NormalTok{(secchi)) }\OperatorTok
\StringTok{  }\KeywordTok{drop_na}\NormalTok{()}
\end{Highlighting}
\end{Shaded}

\begin{enumerate}
\def\labelenumi{\arabic{enumi}.}
\setcounter{enumi}{5}
\tightlist
\item
  Create a plot of mean secchi depth for lakes in Maine, with mean
  secchi depth designated as color and the lake area as the size of the
  dot. Remember that you are using size in the aesthetics and should
  remove the size = 1 from the other part of the code. Adjust the
  transparency of points as needed.
\end{enumerate}

\begin{Shaded}
\begin{Highlighting}[]
\NormalTok{secchi.summary.plot <-}\StringTok{ }\NormalTok{LAGOScombined.maine }\OperatorTok
\StringTok{  }\KeywordTok{group_by}\NormalTok{(lagoslakeid) }\OperatorTok
\StringTok{  }\KeywordTok{summarise}\NormalTok{(}\DataTypeTok{secchi.mean =} \KeywordTok{mean}\NormalTok{(secchi),}
            \DataTypeTok{area =} \KeywordTok{mean}\NormalTok{(lake_area_ha),}
            \DataTypeTok{depth =} \KeywordTok{mean}\NormalTok{(maxdepth),}
            \DataTypeTok{lat =} \KeywordTok{mean}\NormalTok{(nhd_lat),}
            \DataTypeTok{long =} \KeywordTok{mean}\NormalTok{(nhd_long)) }\OperatorTok
\StringTok{  }\KeywordTok{drop_na}\NormalTok{()}

\NormalTok{secchi.sf <-}\StringTok{ }\KeywordTok{st_as_sf}\NormalTok{(secchi.summary.plot, }\DataTypeTok{coords =} \KeywordTok{c}\NormalTok{(}\StringTok{"long"}\NormalTok{, }\StringTok{"lat"}\NormalTok{), }\DataTypeTok{crs =} \DecValTok{4326}\NormalTok{)}

\NormalTok{Secchiplot <-}\StringTok{ }\KeywordTok{ggplot}\NormalTok{() }\OperatorTok{+}
\StringTok{  }\KeywordTok{geom_sf}\NormalTok{(}\DataTypeTok{data =}\NormalTok{ states.subset, }\DataTypeTok{fill =} \StringTok{"white"}\NormalTok{) }\OperatorTok{+}
\StringTok{  }\KeywordTok{geom_sf}\NormalTok{(}\DataTypeTok{data =}\NormalTok{ secchi.sf, }\KeywordTok{aes}\NormalTok{(}\DataTypeTok{color =}\NormalTok{ secchi.mean, }\DataTypeTok{size =}\NormalTok{ area), }
          \DataTypeTok{alpha =} \FloatTok{0.5}\NormalTok{) }\OperatorTok{+}
\StringTok{  }\KeywordTok{scale_color_viridis_c}\NormalTok{() }\OperatorTok{+}
\StringTok{  }\KeywordTok{labs}\NormalTok{(}\DataTypeTok{color =} \StringTok{"Average Secchi Depth (m)"}\NormalTok{, }\DataTypeTok{size =} \StringTok{"Lake Area (ha)"}\NormalTok{) }\OperatorTok{+}
\StringTok{  }\KeywordTok{theme}\NormalTok{(}\DataTypeTok{legend.position =} \StringTok{"top"}\NormalTok{)}
\KeywordTok{print}\NormalTok{(Secchiplot)}
\end{Highlighting}
\end{Shaded}

\includegraphics{A08_Raby_Mapping_files/figure-latex/unnamed-chunk-2-1.pdf}

\begin{enumerate}
\def\labelenumi{\arabic{enumi}.}
\setcounter{enumi}{6}
\tightlist
\item
  Create a second plot, but this time use maximum depth of the lake as
  the size of the dot.
\end{enumerate}

\begin{Shaded}
\begin{Highlighting}[]
\NormalTok{Secchiplot2 <-}\StringTok{ }\KeywordTok{ggplot}\NormalTok{() }\OperatorTok{+}
\StringTok{  }\KeywordTok{geom_sf}\NormalTok{(}\DataTypeTok{data =}\NormalTok{ states.subset, }\DataTypeTok{fill =} \StringTok{"white"}\NormalTok{) }\OperatorTok{+}
\StringTok{  }\KeywordTok{geom_sf}\NormalTok{(}\DataTypeTok{data =}\NormalTok{ secchi.sf, }\KeywordTok{aes}\NormalTok{(}\DataTypeTok{color =}\NormalTok{ secchi.mean, }\DataTypeTok{size =}\NormalTok{ depth), }
          \DataTypeTok{alpha =} \FloatTok{0.5}\NormalTok{) }\OperatorTok{+}
\StringTok{  }\KeywordTok{scale_color_viridis_c}\NormalTok{() }\OperatorTok{+}
\StringTok{  }\KeywordTok{labs}\NormalTok{(}\DataTypeTok{color =} \StringTok{"Average Secchi Depth (m)"}\NormalTok{, }\DataTypeTok{size =} \StringTok{"Maximum Depth (m)"}\NormalTok{) }\OperatorTok{+}
\StringTok{  }\KeywordTok{theme}\NormalTok{(}\DataTypeTok{legend.position =} \StringTok{"top"}\NormalTok{)}
\KeywordTok{print}\NormalTok{(Secchiplot2)}
\end{Highlighting}
\end{Shaded}

\includegraphics{A08_Raby_Mapping_files/figure-latex/unnamed-chunk-3-1.pdf}

\begin{enumerate}
\def\labelenumi{\arabic{enumi}.}
\setcounter{enumi}{7}
\tightlist
\item
  Plot these maps in the same plot with the \texttt{plot\_grid}
  function. Don't worry about adjusting the legends (if you have extra
  time this would be a good bonus task).
\end{enumerate}

\begin{Shaded}
\begin{Highlighting}[]
\KeywordTok{plot_grid}\NormalTok{(Secchiplot, Secchiplot2)}
\end{Highlighting}
\end{Shaded}

\includegraphics{A08_Raby_Mapping_files/figure-latex/unnamed-chunk-4-1.pdf}

What would you change about the legend to make it a more effective
visualization?

\begin{quote}
First of all I would made them fit into the width of the page, so It can
be seen all the information that the legend is suppose to provide.
Second, I would try to create a legend that explains what the actual
diameters of the points mean in terms of Lake Area for one plot and
Maximum Depth for the other one instead of the squares that are been
shown right now. This squares do not communicate anything to the reader
except maybe information about the scale of the values (in the case of
Lake Area it does not even do that, showing only values of thousands of
hectares when many lakes have areas below 100 hectares). You could also
use only one Average Secchi Depth legend after making sure that the
scales in both plots for Secchi Depth is equivalent.
\end{quote}

\begin{enumerate}
\def\labelenumi{\arabic{enumi}.}
\setcounter{enumi}{8}
\tightlist
\item
  What relationships do you see between secchi depth, lake area, and
  lake depth? Which of the two lake variables seems to be a stronger
  determinant of secchi depth? (make a scatterplot and run a regression
  to test this)
\end{enumerate}

\emph{Note: consider log-transforming a predictor variable if
appropriate}

\begin{Shaded}
\begin{Highlighting}[]
\KeywordTok{ggplot}\NormalTok{(secchi.summary.plot, }\KeywordTok{aes}\NormalTok{(}\DataTypeTok{x =} \KeywordTok{log}\NormalTok{(area), }\DataTypeTok{y =}\NormalTok{ secchi.mean)) }\OperatorTok{+}
\StringTok{  }\KeywordTok{geom_point}\NormalTok{() }\OperatorTok{+}
\StringTok{  }\KeywordTok{geom_smooth}\NormalTok{(}\DataTypeTok{method=}\NormalTok{lm) }\OperatorTok{+}
\StringTok{  }\KeywordTok{xlab}\NormalTok{(}\KeywordTok{expression}\NormalTok{(}\StringTok{"log(Area (ha))"}\NormalTok{)) }\OperatorTok{+}
\StringTok{  }\KeywordTok{ylab}\NormalTok{(}\KeywordTok{expression}\NormalTok{(}\StringTok{"Secchi Depth (m)"}\NormalTok{)) }\OperatorTok{+}
\StringTok{  }\KeywordTok{ggtitle}\NormalTok{(}\StringTok{"Lake Area vs Secchi Depth Scatterplot"}\NormalTok{) }\OperatorTok{+}
\StringTok{  }\KeywordTok{theme}\NormalTok{(}\DataTypeTok{plot.title =} \KeywordTok{element_text}\NormalTok{(}\DataTypeTok{hjust =} \FloatTok{0.5}\NormalTok{))}
\end{Highlighting}
\end{Shaded}

\includegraphics{A08_Raby_Mapping_files/figure-latex/unnamed-chunk-5-1.pdf}

\begin{Shaded}
\begin{Highlighting}[]
\NormalTok{lmAreaSecchi <-}\StringTok{ }\KeywordTok{lm}\NormalTok{(}\KeywordTok{log}\NormalTok{(area) }\OperatorTok{~}\StringTok{ }\NormalTok{secchi.mean, }\DataTypeTok{data =}\NormalTok{ secchi.summary.plot)}
\KeywordTok{summary}\NormalTok{(lmAreaSecchi)}
\end{Highlighting}
\end{Shaded}

\begin{verbatim}
## 
## Call:
## lm(formula = log(area) ~ secchi.mean, data = secchi.summary.plot)
## 
## Residuals:
##     Min      1Q  Median      3Q     Max 
## -3.8252 -1.1238 -0.1443  0.9638  5.4036 
## 
## Coefficients:
##             Estimate Std. Error t value Pr(>|t|)    
## (Intercept)  3.19817    0.15107  21.170  < 2e-16 ***
## secchi.mean  0.18865    0.03001   6.287 6.62e-10 ***
## ---
## Signif. codes:  0 '***' 0.001 '**' 0.01 '*' 0.05 '.' 0.1 ' ' 1
## 
## Residual standard error: 1.533 on 547 degrees of freedom
## Multiple R-squared:  0.06739,    Adjusted R-squared:  0.06569 
## F-statistic: 39.53 on 1 and 547 DF,  p-value: 6.622e-10
\end{verbatim}

\begin{quote}
For the Lake Area variable we performed a log transformation because the
tail was too long and most of the values were concentrated in the
smaller area values. In the Lake Area vs Secchi Depth plot it can be
seen that it seems to exists a positive correlation between log(Area)
and Secchi Depth, but with an important variability.
\end{quote}

\begin{quote}
The linear regression confirms these observations. The null hypothesis
that there is no effect of Lake Area on Secchi Depth is rejected with a
5\% level of significance ( F-statistic: 39.53 on 1 and 547 DF, p-value:
6.622e-10 with a positive slope of the regression line = 0.18865);
nevertheless, the linear model explains only 6.5\% of the variance of
the response variable.
\end{quote}

\begin{Shaded}
\begin{Highlighting}[]
\KeywordTok{ggplot}\NormalTok{(secchi.summary.plot, }\KeywordTok{aes}\NormalTok{(}\DataTypeTok{x =}\NormalTok{ depth, }\DataTypeTok{y =}\NormalTok{ secchi.mean)) }\OperatorTok{+}
\StringTok{  }\KeywordTok{geom_point}\NormalTok{() }\OperatorTok{+}
\StringTok{  }\KeywordTok{geom_smooth}\NormalTok{(}\DataTypeTok{method=}\NormalTok{lm) }\OperatorTok{+}
\StringTok{  }\KeywordTok{xlab}\NormalTok{(}\KeywordTok{expression}\NormalTok{(}\StringTok{"Lake Max. Depth (m)"}\NormalTok{)) }\OperatorTok{+}
\StringTok{  }\KeywordTok{ylab}\NormalTok{(}\KeywordTok{expression}\NormalTok{(}\StringTok{"Secchi Depth (m)"}\NormalTok{)) }\OperatorTok{+}
\StringTok{  }\KeywordTok{ggtitle}\NormalTok{(}\StringTok{"Lake Max. Depth vs Secchi Depth Scatterplot"}\NormalTok{) }\OperatorTok{+}
\StringTok{  }\KeywordTok{theme}\NormalTok{(}\DataTypeTok{plot.title =} \KeywordTok{element_text}\NormalTok{(}\DataTypeTok{hjust =} \FloatTok{0.5}\NormalTok{))}
\end{Highlighting}
\end{Shaded}

\includegraphics{A08_Raby_Mapping_files/figure-latex/unnamed-chunk-6-1.pdf}

\begin{Shaded}
\begin{Highlighting}[]
\NormalTok{lmDepthSecchi <-}\StringTok{ }\KeywordTok{lm}\NormalTok{(depth }\OperatorTok{~}\StringTok{ }\NormalTok{secchi.mean, }\DataTypeTok{data =}\NormalTok{ secchi.summary.plot)}
\KeywordTok{summary}\NormalTok{(lmDepthSecchi)}
\end{Highlighting}
\end{Shaded}

\begin{verbatim}
## 
## Call:
## lm(formula = depth ~ secchi.mean, data = secchi.summary.plot)
## 
## Residuals:
##     Min      1Q  Median      3Q     Max 
## -25.971  -4.605  -1.679   2.311  41.476 
## 
## Coefficients:
##             Estimate Std. Error t value Pr(>|t|)    
## (Intercept)  -0.6005     0.7671  -0.783    0.434    
## secchi.mean   2.8548     0.1524  18.738   <2e-16 ***
## ---
## Signif. codes:  0 '***' 0.001 '**' 0.01 '*' 0.05 '.' 0.1 ' ' 1
## 
## Residual standard error: 7.784 on 547 degrees of freedom
## Multiple R-squared:  0.3909, Adjusted R-squared:  0.3898 
## F-statistic: 351.1 on 1 and 547 DF,  p-value: < 2.2e-16
\end{verbatim}

\begin{quote}
In the Lake Maximum Depth vs Secchi Depth plot it can be seen that there
is a clear positive correlation between Lake Max. Depth and Secchi
Depth, and with less variability than Lake Area.
\end{quote}

\begin{quote}
The linear regression confirms these observations. The null hypothesis
that there is no effect of Lake Maximum Depth on Secchi Depth is
rejected with a 5\% level of significance ( F-statistic: 351.1 on 1 and
547 DF, p-value: \textless{} 2.2e-16 with a positive slope of the
regression line = 2.8548); moreover, the linear model explains 39\% of
the variance of the response variable.
\end{quote}

\hypertarget{mapping-water-features-and-watershed-boundaries}{%
\subsection{Mapping water features and watershed
boundaries}\label{mapping-water-features-and-watershed-boundaries}}

\begin{enumerate}
\def\labelenumi{\arabic{enumi}.}
\setcounter{enumi}{9}
\tightlist
\item
  Wrangle the USA rivers and HUC6 watershed boundaries dataset so that
  they include only the features present in Florida (FL). Adjust the
  coordinate reference systems if necessary to ensure they use the same
  projection.
\end{enumerate}

\begin{Shaded}
\begin{Highlighting}[]
\NormalTok{HUC6.FL <-}\StringTok{ }\NormalTok{HUC6.WS }\OperatorTok
\StringTok{  }\KeywordTok{filter}\NormalTok{(}\KeywordTok{grepl}\NormalTok{(}\StringTok{"FL"}\NormalTok{, States))}

\CommentTok{#To chech that it worked}
\KeywordTok{summary}\NormalTok{(HUC6.FL}\OperatorTok{$}\NormalTok{States)}
\end{Highlighting}
\end{Shaded}

\begin{verbatim}
##       AL    AL,FL AL,FL,GA AL,GA,TN AL,LA,MS    AL,MS       FL    FL,GA 
##        0        3        1        0        0        0        6        4 
##       GA GA,NC,SC    LA,MS       NC    NC,SC NC,SC,VA    NC,VA       SC 
##        0        0        0        0        0        0        0        0
\end{verbatim}

\begin{Shaded}
\begin{Highlighting}[]
\CommentTok{#it did}

\CommentTok{# Filter for Florida}
\NormalTok{USAwaterfeatures <-}\StringTok{ }\KeywordTok{filter}\NormalTok{(USAwaterfeatures, STATE }\OperatorTok{==}\StringTok{ "FL"}\NormalTok{)}

\CommentTok{# Remove a couple feature types we don't care about}
\NormalTok{USAwaterfeatures <-}\StringTok{ }\KeywordTok{filter}\NormalTok{(USAwaterfeatures, FEATURE }\OperatorTok{!=}\StringTok{ "Apparent Limit"} \OperatorTok{&}\StringTok{ }\NormalTok{FEATURE }\OperatorTok{!=}\StringTok{ "Closure Line"}\NormalTok{)}

\KeywordTok{st_crs}\NormalTok{(USAwaterfeatures)}
\end{Highlighting}
\end{Shaded}

\begin{verbatim}
## Coordinate Reference System: NA
\end{verbatim}

\begin{Shaded}
\begin{Highlighting}[]
\KeywordTok{st_crs}\NormalTok{(HUC6.FL)}
\end{Highlighting}
\end{Shaded}

\begin{verbatim}
## Coordinate Reference System:
##   EPSG: 4269 
##   proj4string: "+proj=longlat +datum=NAD83 +no_defs"
\end{verbatim}

\begin{Shaded}
\begin{Highlighting}[]
\CommentTok{# USAwaterfeatures does not have coordinate reference information. HUC6 coordinate}
\CommentTok{# reference information is NAD83 ESPG: 4269.}

\NormalTok{USAwaterfeatures <-}\StringTok{ }\KeywordTok{st_set_crs}\NormalTok{(USAwaterfeatures, }\DecValTok{4269}\NormalTok{)}
\KeywordTok{st_crs}\NormalTok{(USAwaterfeatures)}
\end{Highlighting}
\end{Shaded}

\begin{verbatim}
## Coordinate Reference System:
##   EPSG: 4269 
##   proj4string: "+proj=longlat +ellps=GRS80 +towgs84=0,0,0,0,0,0,0 +no_defs"
\end{verbatim}

\begin{enumerate}
\def\labelenumi{\arabic{enumi}.}
\setcounter{enumi}{10}
\tightlist
\item
  Create a map of watershed boundaries in Florida, with the layer of
  water features on top. Color the watersheds gray (make sure the lines
  separating watersheds are still visible) and color the water features
  by type.
\end{enumerate}

\begin{Shaded}
\begin{Highlighting}[]
\NormalTok{FLlayers <-}\StringTok{ }\KeywordTok{ggplot}\NormalTok{() }\OperatorTok{+}
\StringTok{  }\KeywordTok{geom_sf}\NormalTok{(}\DataTypeTok{data =}\NormalTok{ HUC6.FL, }\DataTypeTok{color =} \StringTok{"gray58"}\NormalTok{, }\DataTypeTok{fill =} \StringTok{"gray80"}\NormalTok{, }\DataTypeTok{alpha =} \FloatTok{0.5}\NormalTok{) }\OperatorTok{+}
\StringTok{  }\KeywordTok{geom_sf}\NormalTok{(}\DataTypeTok{data =}\NormalTok{ USAwaterfeatures, }\KeywordTok{aes}\NormalTok{(}\DataTypeTok{fill =}\NormalTok{ FEATURE, }\DataTypeTok{color =}\NormalTok{ FEATURE)) }\OperatorTok{+}
\StringTok{  }\KeywordTok{scale_fill_brewer}\NormalTok{(}\DataTypeTok{palette =} \StringTok{"Set1"}\NormalTok{) }\OperatorTok{+}
\StringTok{  }\KeywordTok{scale_color_brewer}\NormalTok{(}\DataTypeTok{palette =} \StringTok{"Set1"}\NormalTok{) }\OperatorTok{+}
\StringTok{  }\KeywordTok{labs}\NormalTok{(}\DataTypeTok{color =} \StringTok{'Water Feature'}\NormalTok{, }\DataTypeTok{fill =} \StringTok{'Water Feature'}\NormalTok{) }\OperatorTok{+}
\StringTok{  }\KeywordTok{ggtitle}\NormalTok{(}\StringTok{"Watershed Boundaries and Water Features in Florida"}\NormalTok{) }\OperatorTok{+}
\StringTok{  }\KeywordTok{theme}\NormalTok{(}\DataTypeTok{plot.title =} \KeywordTok{element_text}\NormalTok{(}\DataTypeTok{hjust =} \FloatTok{0.5}\NormalTok{))}
\KeywordTok{print}\NormalTok{(FLlayers)}
\end{Highlighting}
\end{Shaded}

\includegraphics{A08_Raby_Mapping_files/figure-latex/unnamed-chunk-8-1.pdf}

\begin{enumerate}
\def\labelenumi{\arabic{enumi}.}
\setcounter{enumi}{11}
\tightlist
\item
  What are the dominant water features in Florida? How does this
  distribution differ (or not) compared to North Carolina?
\end{enumerate}

\begin{quote}
In most of the inland part of the state the most dominant features are
streams and lakes; nevertheless, in the south part canals are clearly
more dominant than streams. Banks can be appreciated all over both
coastal lines. The extreme south part of the states show some
intracoastal waterways. Meanwhile, North Carolina is almost everywhere
dominated by streams and by man made reservoirs built in the stream's
riverbeds. There are some canals near the coast but are far less
dominant than canals in south Florida.
\end{quote}

\hypertarget{reflection}{%
\subsection{Reflection}\label{reflection}}

\begin{enumerate}
\def\labelenumi{\arabic{enumi}.}
\setcounter{enumi}{12}
\tightlist
\item
  What are 2-3 conclusions or summary points about mapping you learned
  through your analysis?
\end{enumerate}

\begin{quote}
\begin{enumerate}
\def\labelenumi{\arabic{enumi}.}
\tightlist
\item
  Lake Area can have a positive correlation with Secchi Depth
\item
  Lake Maximum Depth can have a positive correlation with Secchi Depth,
  greater than Lake Area and explaining more variance.
\item
  Dominant water features differ among states.
\end{enumerate}
\end{quote}

\begin{enumerate}
\def\labelenumi{\arabic{enumi}.}
\setcounter{enumi}{13}
\tightlist
\item
  What data, visualizations, and/or models supported your conclusions
  from 13?
\end{enumerate}

\begin{quote}
\begin{enumerate}
\def\labelenumi{\arabic{enumi}.}
\tightlist
\item
  Log(Lake Area) and Secchi Depth scatterplot and linear model.
\item
  Lake Maximum Depth and Secchi Depth scatterplot and linear model.
\item
  Maps of water features created for Florida and NC (this last one in
  class).
\end{enumerate}
\end{quote}

\begin{enumerate}
\def\labelenumi{\arabic{enumi}.}
\setcounter{enumi}{14}
\tightlist
\item
  Did hands-on data analysis impact your learning about mapping relative
  to a theory-based lesson? If so, how?
\end{enumerate}

\begin{quote}
It did. Playing with data is the best way to understand concepts and
later have them available to apply them in other settings.
\end{quote}

\begin{enumerate}
\def\labelenumi{\arabic{enumi}.}
\setcounter{enumi}{15}
\tightlist
\item
  How did the real-world data compare with your expectations from
  theory?
\end{enumerate}

\begin{quote}
I did not have much expectations from theory in this area.
\end{quote}


\end{document}
