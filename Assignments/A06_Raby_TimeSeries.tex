\documentclass[]{article}
\usepackage{lmodern}
\usepackage{amssymb,amsmath}
\usepackage{ifxetex,ifluatex}
\usepackage{fixltx2e} % provides \textsubscript
\ifnum 0\ifxetex 1\fi\ifluatex 1\fi=0 % if pdftex
  \usepackage[T1]{fontenc}
  \usepackage[utf8]{inputenc}
\else % if luatex or xelatex
  \ifxetex
    \usepackage{mathspec}
  \else
    \usepackage{fontspec}
  \fi
  \defaultfontfeatures{Ligatures=TeX,Scale=MatchLowercase}
\fi
% use upquote if available, for straight quotes in verbatim environments
\IfFileExists{upquote.sty}{\usepackage{upquote}}{}
% use microtype if available
\IfFileExists{microtype.sty}{%
\usepackage{microtype}
\UseMicrotypeSet[protrusion]{basicmath} % disable protrusion for tt fonts
}{}
\usepackage[margin=2.54cm]{geometry}
\usepackage{hyperref}
\hypersetup{unicode=true,
            pdftitle={Assignment 6: Time Series Analysis},
            pdfauthor={Felipe Raby Amadori},
            pdfborder={0 0 0},
            breaklinks=true}
\urlstyle{same}  % don't use monospace font for urls
\usepackage{color}
\usepackage{fancyvrb}
\newcommand{\VerbBar}{|}
\newcommand{\VERB}{\Verb[commandchars=\\\{\}]}
\DefineVerbatimEnvironment{Highlighting}{Verbatim}{commandchars=\\\{\}}
% Add ',fontsize=\small' for more characters per line
\usepackage{framed}
\definecolor{shadecolor}{RGB}{248,248,248}
\newenvironment{Shaded}{\begin{snugshade}}{\end{snugshade}}
\newcommand{\AlertTok}[1]{\textcolor[rgb]{0.94,0.16,0.16}{#1}}
\newcommand{\AnnotationTok}[1]{\textcolor[rgb]{0.56,0.35,0.01}{\textbf{\textit{#1}}}}
\newcommand{\AttributeTok}[1]{\textcolor[rgb]{0.77,0.63,0.00}{#1}}
\newcommand{\BaseNTok}[1]{\textcolor[rgb]{0.00,0.00,0.81}{#1}}
\newcommand{\BuiltInTok}[1]{#1}
\newcommand{\CharTok}[1]{\textcolor[rgb]{0.31,0.60,0.02}{#1}}
\newcommand{\CommentTok}[1]{\textcolor[rgb]{0.56,0.35,0.01}{\textit{#1}}}
\newcommand{\CommentVarTok}[1]{\textcolor[rgb]{0.56,0.35,0.01}{\textbf{\textit{#1}}}}
\newcommand{\ConstantTok}[1]{\textcolor[rgb]{0.00,0.00,0.00}{#1}}
\newcommand{\ControlFlowTok}[1]{\textcolor[rgb]{0.13,0.29,0.53}{\textbf{#1}}}
\newcommand{\DataTypeTok}[1]{\textcolor[rgb]{0.13,0.29,0.53}{#1}}
\newcommand{\DecValTok}[1]{\textcolor[rgb]{0.00,0.00,0.81}{#1}}
\newcommand{\DocumentationTok}[1]{\textcolor[rgb]{0.56,0.35,0.01}{\textbf{\textit{#1}}}}
\newcommand{\ErrorTok}[1]{\textcolor[rgb]{0.64,0.00,0.00}{\textbf{#1}}}
\newcommand{\ExtensionTok}[1]{#1}
\newcommand{\FloatTok}[1]{\textcolor[rgb]{0.00,0.00,0.81}{#1}}
\newcommand{\FunctionTok}[1]{\textcolor[rgb]{0.00,0.00,0.00}{#1}}
\newcommand{\ImportTok}[1]{#1}
\newcommand{\InformationTok}[1]{\textcolor[rgb]{0.56,0.35,0.01}{\textbf{\textit{#1}}}}
\newcommand{\KeywordTok}[1]{\textcolor[rgb]{0.13,0.29,0.53}{\textbf{#1}}}
\newcommand{\NormalTok}[1]{#1}
\newcommand{\OperatorTok}[1]{\textcolor[rgb]{0.81,0.36,0.00}{\textbf{#1}}}
\newcommand{\OtherTok}[1]{\textcolor[rgb]{0.56,0.35,0.01}{#1}}
\newcommand{\PreprocessorTok}[1]{\textcolor[rgb]{0.56,0.35,0.01}{\textit{#1}}}
\newcommand{\RegionMarkerTok}[1]{#1}
\newcommand{\SpecialCharTok}[1]{\textcolor[rgb]{0.00,0.00,0.00}{#1}}
\newcommand{\SpecialStringTok}[1]{\textcolor[rgb]{0.31,0.60,0.02}{#1}}
\newcommand{\StringTok}[1]{\textcolor[rgb]{0.31,0.60,0.02}{#1}}
\newcommand{\VariableTok}[1]{\textcolor[rgb]{0.00,0.00,0.00}{#1}}
\newcommand{\VerbatimStringTok}[1]{\textcolor[rgb]{0.31,0.60,0.02}{#1}}
\newcommand{\WarningTok}[1]{\textcolor[rgb]{0.56,0.35,0.01}{\textbf{\textit{#1}}}}
\usepackage{graphicx,grffile}
\makeatletter
\def\maxwidth{\ifdim\Gin@nat@width>\linewidth\linewidth\else\Gin@nat@width\fi}
\def\maxheight{\ifdim\Gin@nat@height>\textheight\textheight\else\Gin@nat@height\fi}
\makeatother
% Scale images if necessary, so that they will not overflow the page
% margins by default, and it is still possible to overwrite the defaults
% using explicit options in \includegraphics[width, height, ...]{}
\setkeys{Gin}{width=\maxwidth,height=\maxheight,keepaspectratio}
\IfFileExists{parskip.sty}{%
\usepackage{parskip}
}{% else
\setlength{\parindent}{0pt}
\setlength{\parskip}{6pt plus 2pt minus 1pt}
}
\setlength{\emergencystretch}{3em}  % prevent overfull lines
\providecommand{\tightlist}{%
  \setlength{\itemsep}{0pt}\setlength{\parskip}{0pt}}
\setcounter{secnumdepth}{0}
% Redefines (sub)paragraphs to behave more like sections
\ifx\paragraph\undefined\else
\let\oldparagraph\paragraph
\renewcommand{\paragraph}[1]{\oldparagraph{#1}\mbox{}}
\fi
\ifx\subparagraph\undefined\else
\let\oldsubparagraph\subparagraph
\renewcommand{\subparagraph}[1]{\oldsubparagraph{#1}\mbox{}}
\fi

%%% Use protect on footnotes to avoid problems with footnotes in titles
\let\rmarkdownfootnote\footnote%
\def\footnote{\protect\rmarkdownfootnote}

%%% Change title format to be more compact
\usepackage{titling}

% Create subtitle command for use in maketitle
\providecommand{\subtitle}[1]{
  \posttitle{
    \begin{center}\large#1\end{center}
    }
}

\setlength{\droptitle}{-2em}

  \title{Assignment 6: Time Series Analysis}
    \pretitle{\vspace{\droptitle}\centering\huge}
  \posttitle{\par}
    \author{Felipe Raby Amadori}
    \preauthor{\centering\large\emph}
  \postauthor{\par}
    \date{}
    \predate{}\postdate{}
  

\begin{document}
\maketitle

\hypertarget{overview}{%
\subsection{OVERVIEW}\label{overview}}

This exercise accompanies the lessons in Hydrologic Data Analysis on
time series analysis

\hypertarget{directions}{%
\subsection{Directions}\label{directions}}

\begin{enumerate}
\def\labelenumi{\arabic{enumi}.}
\tightlist
\item
  Change ``Student Name'' on line 3 (above) with your name.
\item
  Work through the steps, \textbf{creating code and output} that fulfill
  each instruction.
\item
  Be sure to \textbf{answer the questions} in this assignment document.
\item
  When you have completed the assignment, \textbf{Knit} the text and
  code into a single pdf file.
\item
  After Knitting, submit the completed exercise (pdf file) to the
  dropbox in Sakai. Add your last name into the file name (e.g.,
  ``A06\_Salk.html'') prior to submission.
\end{enumerate}

The completed exercise is due on 11 October 2019 at 9:00 am.

\hypertarget{setup}{%
\subsection{Setup}\label{setup}}

\begin{enumerate}
\def\labelenumi{\arabic{enumi}.}
\tightlist
\item
  Verify your working directory is set to the R project file,
\item
  Load the tidyverse, lubridate, trend, and dataRetrieval packages.
\item
  Set your ggplot theme (can be theme\_classic or something else)
\item
  Load the ClearCreekDischarge.Monthly.csv file from the processed data
  folder. Call this data frame ClearCreekDischarge.Monthly.
\end{enumerate}

\begin{Shaded}
\begin{Highlighting}[]
\NormalTok{knitr}\OperatorTok{::}\NormalTok{opts_chunk}\OperatorTok{$}\KeywordTok{set}\NormalTok{(}\DataTypeTok{message =} \OtherTok{FALSE}\NormalTok{, }\DataTypeTok{warning =} \OtherTok{FALSE}\NormalTok{)}

\CommentTok{#Verify your working directory is set to the R project file}
\KeywordTok{getwd}\NormalTok{()}
\end{Highlighting}
\end{Shaded}

\begin{verbatim}
## [1] "C:/Users/Felipe/OneDrive - Duke University/1. DUKE/Ramos 3 Semestre/Hydrologic_Data_Analysis"
\end{verbatim}

\begin{Shaded}
\begin{Highlighting}[]
\CommentTok{#Load the tidyverse, lubridate, trend, and dataRetrieval packages}
\KeywordTok{library}\NormalTok{(tidyverse)}
\end{Highlighting}
\end{Shaded}

\begin{verbatim}
## -- Attaching packages --------------------------------------------------------------------------------------- tidyverse 1.2.1 --
\end{verbatim}

\begin{verbatim}
## v ggplot2 3.2.1     v purrr   0.3.2
## v tibble  2.1.3     v dplyr   0.8.3
## v tidyr   0.8.3     v stringr 1.4.0
## v readr   1.3.1     v forcats 0.4.0
\end{verbatim}

\begin{verbatim}
## -- Conflicts ------------------------------------------------------------------------------------------ tidyverse_conflicts() --
## x dplyr::filter() masks stats::filter()
## x dplyr::lag()    masks stats::lag()
\end{verbatim}

\begin{Shaded}
\begin{Highlighting}[]
\KeywordTok{library}\NormalTok{(trend)}
\KeywordTok{library}\NormalTok{(dataRetrieval)}
\KeywordTok{library}\NormalTok{(lubridate)}
\end{Highlighting}
\end{Shaded}

\begin{verbatim}
## 
## Attaching package: 'lubridate'
\end{verbatim}

\begin{verbatim}
## The following object is masked from 'package:base':
## 
##     date
\end{verbatim}

\begin{Shaded}
\begin{Highlighting}[]
\KeywordTok{library}\NormalTok{(ggplot2)}

\CommentTok{#Set your ggplot theme}
\NormalTok{felipe_theme <-}\StringTok{ }\KeywordTok{theme_light}\NormalTok{(}\DataTypeTok{base_size =} \DecValTok{12}\NormalTok{) }\OperatorTok{+}
\StringTok{  }\KeywordTok{theme}\NormalTok{(}\DataTypeTok{axis.text =} \KeywordTok{element_text}\NormalTok{(}\DataTypeTok{color =} \StringTok{"grey8"}\NormalTok{), }
        \DataTypeTok{legend.position =} \StringTok{"right"}\NormalTok{, }\DataTypeTok{plot.title =} \KeywordTok{element_text}\NormalTok{(}\DataTypeTok{hjust =} \FloatTok{0.5}\NormalTok{)) }
\KeywordTok{theme_set}\NormalTok{(felipe_theme)}


\CommentTok{#Load the ClearCreekDischarge.Monthly.csv file from the processed data folder.}
\NormalTok{ClearCreekDischarge.Monthly <-}\StringTok{ }\KeywordTok{read.csv}\NormalTok{(}\StringTok{"./Data/Processed/ClearCreekDischarge.Monthly.csv"}\NormalTok{)}
\end{Highlighting}
\end{Shaded}

\hypertarget{time-series-decomposition}{%
\subsection{Time Series Decomposition}\label{time-series-decomposition}}

\begin{enumerate}
\def\labelenumi{\arabic{enumi}.}
\setcounter{enumi}{4}
\tightlist
\item
  Create a new data frame that includes daily mean discharge at the Eno
  River for all available dates (\texttt{siteNumbers\ =\ "02085070"}).
  Rename the columns accordingly.
\item
  Plot discharge over time with geom\_line. Make sure axis labels are
  formatted appropriately.
\item
  Create a time series of discharge
\item
  Decompose the time series using the \texttt{stl} function.
\item
  Visualize the decomposed time series.
\end{enumerate}

\begin{Shaded}
\begin{Highlighting}[]
\CommentTok{# Create a new data frame that includes daily mean discharge at the Eno River for all}
\CommentTok{# available dates (`siteNumbers = "02085070"`). Rename the columns accordingly.}
\NormalTok{EnoRiverDischarge <-}\StringTok{ }\KeywordTok{readNWISdv}\NormalTok{(}\DataTypeTok{siteNumbers =} \StringTok{"02085070"}\NormalTok{,}
                     \DataTypeTok{parameterCd =} \StringTok{"00060"}\NormalTok{, }\CommentTok{# discharge (ft3/s)}
                     \DataTypeTok{startDate =} \StringTok{""}\NormalTok{,}
                     \DataTypeTok{endDate =} \StringTok{""}\NormalTok{)}
\KeywordTok{names}\NormalTok{(EnoRiverDischarge)[}\DecValTok{4}\OperatorTok{:}\DecValTok{5}\NormalTok{] <-}\StringTok{ }\KeywordTok{c}\NormalTok{(}\StringTok{"Discharge"}\NormalTok{, }\StringTok{"Approval.Code"}\NormalTok{)}

\CommentTok{# Plot discharge over time with geom_line. Make sure axis labels are formatted appropriately.}
\NormalTok{EnoRiver.plot <-}\StringTok{ }\KeywordTok{ggplot}\NormalTok{(EnoRiverDischarge, }\KeywordTok{aes}\NormalTok{(}\DataTypeTok{x =}\NormalTok{ Date)) }\OperatorTok{+}
\StringTok{  }\KeywordTok{geom_line}\NormalTok{(}\KeywordTok{aes}\NormalTok{(}\DataTypeTok{y =}\NormalTok{ Discharge), }\DataTypeTok{color =} \StringTok{"blue"}\NormalTok{, }\DataTypeTok{show.legend =} \OtherTok{FALSE}\NormalTok{) }\OperatorTok{+}
\StringTok{  }\KeywordTok{labs}\NormalTok{(}\DataTypeTok{x =} \StringTok{""}\NormalTok{, }\DataTypeTok{y =} \KeywordTok{expression}\NormalTok{(}\StringTok{"Discharge (ft"}\OperatorTok{^}\DecValTok{3}\OperatorTok{*}\StringTok{"/s)"}\NormalTok{), }
       \DataTypeTok{title =} \StringTok{"Eno River Discharge Over Time"}\NormalTok{) }\OperatorTok{+}
\StringTok{  }\KeywordTok{theme}\NormalTok{(}\DataTypeTok{plot.title =} \KeywordTok{element_text}\NormalTok{(}\DataTypeTok{size =} \DecValTok{12}\NormalTok{),}
        \DataTypeTok{axis.title.x =} \KeywordTok{element_blank}\NormalTok{())}

\KeywordTok{print}\NormalTok{(EnoRiver.plot)}
\end{Highlighting}
\end{Shaded}

\includegraphics{A06_Raby_TimeSeries_files/figure-latex/unnamed-chunk-1-1.pdf}

\begin{Shaded}
\begin{Highlighting}[]
\CommentTok{# Create a time series of discharge}

\CommentTok{# First we check if there are gaps in the data.}
\KeywordTok{table}\NormalTok{(}\KeywordTok{diff}\NormalTok{(EnoRiverDischarge}\OperatorTok{$}\NormalTok{Date))}
\end{Highlighting}
\end{Shaded}

\begin{verbatim}
## 
##     1    39 
## 20449     1
\end{verbatim}

\begin{Shaded}
\begin{Highlighting}[]
\CommentTok{#There is 1 data gap of 38 measurements}

\CommentTok{# Looking for the dates where there is no measurement}
\NormalTok{DateRange <-}\StringTok{ }\KeywordTok{seq}\NormalTok{(}\KeywordTok{min}\NormalTok{(EnoRiverDischarge}\OperatorTok{$}\NormalTok{Date), }\KeywordTok{max}\NormalTok{(EnoRiverDischarge}\OperatorTok{$}\NormalTok{Date), }\DataTypeTok{by =} \DecValTok{1}\NormalTok{)}
\NormalTok{DateRange[}\OperatorTok{!}\NormalTok{DateRange }\OperatorTok\StringTok{ }\NormalTok{EnoRiverDischarge}\OperatorTok{$}\NormalTok{Date]}
\end{Highlighting}
\end{Shaded}

\begin{verbatim}
##  [1] "2017-10-21" "2017-10-22" "2017-10-23" "2017-10-24" "2017-10-25"
##  [6] "2017-10-26" "2017-10-27" "2017-10-28" "2017-10-29" "2017-10-30"
## [11] "2017-10-31" "2017-11-01" "2017-11-02" "2017-11-03" "2017-11-04"
## [16] "2017-11-05" "2017-11-06" "2017-11-07" "2017-11-08" "2017-11-09"
## [21] "2017-11-10" "2017-11-11" "2017-11-12" "2017-11-13" "2017-11-14"
## [26] "2017-11-15" "2017-11-16" "2017-11-17" "2017-11-18" "2017-11-19"
## [31] "2017-11-20" "2017-11-21" "2017-11-22" "2017-11-23" "2017-11-24"
## [36] "2017-11-25" "2017-11-26" "2017-11-27"
\end{verbatim}

\begin{Shaded}
\begin{Highlighting}[]
\CommentTok{#Data gap between 10/21/2017 & 11/27/2017}

\CommentTok{#Considering that time series objects must be equispaced, that the gap is to big to interpolate}
\CommentTok{#using the neighboring values, and that # after 11/27/2017 we only have less than two years of data}
\CommentTok{#of an approx. 56 years dataset, we are only going to consider for the analysis }
\CommentTok{#the data from 09/01/1963 to 10/20/2017.}

\CommentTok{#Filtering the data}
\NormalTok{EnoRiverDischarge2 <-}\StringTok{ }\KeywordTok{filter}\NormalTok{(EnoRiverDischarge, Date }\OperatorTok{<}\StringTok{ "2017-10-21"}\NormalTok{)}

\CommentTok{#Checking for gaps}
\KeywordTok{table}\NormalTok{(}\KeywordTok{diff}\NormalTok{(EnoRiverDischarge2}\OperatorTok{$}\NormalTok{Date))}
\end{Highlighting}
\end{Shaded}

\begin{verbatim}
## 
##     1 
## 19773
\end{verbatim}

\begin{Shaded}
\begin{Highlighting}[]
\CommentTok{#Plot of the selected data}
\NormalTok{EnoRiver.plot2 <-}\StringTok{ }\KeywordTok{ggplot}\NormalTok{(EnoRiverDischarge2, }\KeywordTok{aes}\NormalTok{(}\DataTypeTok{x =}\NormalTok{ Date)) }\OperatorTok{+}
\StringTok{  }\KeywordTok{geom_line}\NormalTok{(}\KeywordTok{aes}\NormalTok{(}\DataTypeTok{y =}\NormalTok{ Discharge), }\DataTypeTok{color =} \StringTok{"blue"}\NormalTok{, }\DataTypeTok{show.legend =} \OtherTok{FALSE}\NormalTok{) }\OperatorTok{+}
\StringTok{  }\KeywordTok{labs}\NormalTok{(}\DataTypeTok{x =} \StringTok{""}\NormalTok{, }\DataTypeTok{y =} \KeywordTok{expression}\NormalTok{(}\StringTok{"Discharge (ft"}\OperatorTok{^}\DecValTok{3}\OperatorTok{*}\StringTok{"/s)"}\NormalTok{), }
       \DataTypeTok{title =} \StringTok{"Eno River Discharge Over Time"}\NormalTok{) }\OperatorTok{+}
\StringTok{  }\KeywordTok{theme}\NormalTok{(}\DataTypeTok{plot.title =} \KeywordTok{element_text}\NormalTok{(}\DataTypeTok{size =} \DecValTok{12}\NormalTok{),}
        \DataTypeTok{axis.title.x =} \KeywordTok{element_blank}\NormalTok{())}

\KeywordTok{print}\NormalTok{(EnoRiver.plot2)}
\end{Highlighting}
\end{Shaded}

\includegraphics{A06_Raby_TimeSeries_files/figure-latex/unnamed-chunk-1-2.pdf}

\begin{Shaded}
\begin{Highlighting}[]
\CommentTok{#Time series of discharge}
\NormalTok{EnoRiver_ts <-}\StringTok{ }\KeywordTok{ts}\NormalTok{(EnoRiverDischarge2[[}\DecValTok{4}\NormalTok{]], }\DataTypeTok{frequency =} \DecValTok{365}\NormalTok{)}

\CommentTok{#Decompose the time series using the `stl` function.}
\NormalTok{EnoRiver_Decomposed <-}\StringTok{ }\KeywordTok{stl}\NormalTok{(EnoRiver_ts, }\DataTypeTok{s.window =} \StringTok{"periodic"}\NormalTok{)}



\CommentTok{#From class 11}
\CommentTok{# Import data}
\NormalTok{ClearCreekDischarge <-}\StringTok{ }\KeywordTok{readNWISdv}\NormalTok{(}\DataTypeTok{siteNumbers =} \StringTok{"06719505"}\NormalTok{,}
                     \DataTypeTok{parameterCd =} \StringTok{"00060"}\NormalTok{, }\CommentTok{# discharge (ft3/s)}
                     \DataTypeTok{startDate =} \StringTok{""}\NormalTok{,}
                     \DataTypeTok{endDate =} \StringTok{""}\NormalTok{)}
\KeywordTok{names}\NormalTok{(ClearCreekDischarge)[}\DecValTok{4}\OperatorTok{:}\DecValTok{5}\NormalTok{] <-}\StringTok{ }\KeywordTok{c}\NormalTok{(}\StringTok{"Discharge"}\NormalTok{, }\StringTok{"Approval.Code"}\NormalTok{)}
\NormalTok{ClearCreek_ts <-}\StringTok{ }\KeywordTok{ts}\NormalTok{(ClearCreekDischarge[[}\DecValTok{4}\NormalTok{]], }\DataTypeTok{frequency =} \DecValTok{365}\NormalTok{)}
\NormalTok{ClearCreek_Decomposed <-}\StringTok{ }\KeywordTok{stl}\NormalTok{(ClearCreek_ts, }\DataTypeTok{s.window =} \StringTok{"periodic"}\NormalTok{)}


\CommentTok{#Visualize the decomposed Eno River time series.}
\KeywordTok{plot}\NormalTok{(EnoRiver_Decomposed)}
\end{Highlighting}
\end{Shaded}

\includegraphics{A06_Raby_TimeSeries_files/figure-latex/unnamed-chunk-1-3.pdf}

\begin{Shaded}
\begin{Highlighting}[]
\CommentTok{# Visualize the decomposed Clear Creek series. }
\KeywordTok{plot}\NormalTok{(ClearCreek_Decomposed)}
\end{Highlighting}
\end{Shaded}

\includegraphics{A06_Raby_TimeSeries_files/figure-latex/unnamed-chunk-1-4.pdf}

\begin{Shaded}
\begin{Highlighting}[]
\CommentTok{# Zooming to check what is the smaller peak in the seasonal plot}
\NormalTok{EnoRiverDischarge3 <-}\StringTok{ }\KeywordTok{filter}\NormalTok{(EnoRiverDischarge, Date }\OperatorTok{<}\StringTok{ "1967-10-21"}\NormalTok{)}
\CommentTok{#Plot of the selected data}
\NormalTok{EnoRiver.plot3 <-}\StringTok{ }\KeywordTok{ggplot}\NormalTok{(EnoRiverDischarge3, }\KeywordTok{aes}\NormalTok{(}\DataTypeTok{x =}\NormalTok{ Date)) }\OperatorTok{+}
\StringTok{  }\KeywordTok{geom_line}\NormalTok{(}\KeywordTok{aes}\NormalTok{(}\DataTypeTok{y =}\NormalTok{ Discharge), }\DataTypeTok{color =} \StringTok{"blue"}\NormalTok{, }\DataTypeTok{show.legend =} \OtherTok{FALSE}\NormalTok{) }\OperatorTok{+}
\StringTok{  }\KeywordTok{labs}\NormalTok{(}\DataTypeTok{x =} \StringTok{""}\NormalTok{, }\DataTypeTok{y =} \KeywordTok{expression}\NormalTok{(}\StringTok{"Discharge (ft"}\OperatorTok{^}\DecValTok{3}\OperatorTok{*}\StringTok{"/s)"}\NormalTok{), }
       \DataTypeTok{title =} \StringTok{"Eno River Discharge Over Time"}\NormalTok{) }\OperatorTok{+}
\StringTok{  }\KeywordTok{theme}\NormalTok{(}\DataTypeTok{plot.title =} \KeywordTok{element_text}\NormalTok{(}\DataTypeTok{size =} \DecValTok{12}\NormalTok{),}
        \DataTypeTok{axis.title.x =} \KeywordTok{element_blank}\NormalTok{())}

\KeywordTok{print}\NormalTok{(EnoRiver.plot3)}
\end{Highlighting}
\end{Shaded}

\includegraphics{A06_Raby_TimeSeries_files/figure-latex/unnamed-chunk-1-5.pdf}

\begin{enumerate}
\def\labelenumi{\arabic{enumi}.}
\setcounter{enumi}{9}
\tightlist
\item
  How do the seasonal and trend components of the decomposition compare
  to the Clear Creek discharge dataset? Are they similar in magnitude?
\end{enumerate}

\#data, seasonal info, trend, remainder \#Trend is all over the place.
\#remainder. All the error that is not predicted by season or trend,
\#the trends coincide with the high discharge. The trend is always
positive. \#There are some negative seasonal values.

\begin{quote}
Seasonal: There is a clear seasonality in the Eno River and the Clear
Creek discharge datasets. Nevertheless, the Eno River data has a double
seasonal peak. It has a smaller peak in between the big peaks. It has
the bigger seasonal peak in the first months of the year and then it has
a smaller seasonal peak in spring/Summer. They are not similar in
magnitude probably because the discharge data of the two rivers is very
different in magnitude as well.
\end{quote}

\begin{quote}
Trend: Both datasets do not show any clear trend. Even though the
magnitude of the Eno River Discharge data is greater than the Clear
Creek Discharge data, the trend plots of both rivers have similar
magnitude; moreover, Clear Creek has higher local trends (values over
300 ft/s) than the Eno River (only 1 peak over 250 ft/s).
\end{quote}

\hypertarget{trend-analysis}{%
\subsection{Trend Analysis}\label{trend-analysis}}

Research question: Has there been a monotonic trend in discharge in
Clear Creek over the period of study?

\begin{enumerate}
\def\labelenumi{\arabic{enumi}.}
\setcounter{enumi}{10}
\tightlist
\item
  Generate a time series of monthly discharge in Clear Creek from the
  ClearCreekDischarge.Monthly data frame. This time series should
  include just one column (discharge).
\item
  Run a Seasonal Mann-Kendall test on the monthly discharge data.
  Inspect the overall trend and the monthly trends.
\end{enumerate}

\begin{Shaded}
\begin{Highlighting}[]
\CommentTok{# The SMK test requires identically distributed data, condition that this dataset complies with.}

\NormalTok{ClearCreekDischarge.Monthly_ts <-}\StringTok{ }\KeywordTok{ts}\NormalTok{(ClearCreekDischarge.Monthly[[}\DecValTok{3}\NormalTok{]], }
                                     \DataTypeTok{frequency =} \DecValTok{12}\NormalTok{, }\DataTypeTok{start =} \KeywordTok{c}\NormalTok{(}\DecValTok{1974}\NormalTok{, }\DecValTok{10}\NormalTok{)) }
\CommentTok{# 12 because we have monthly data.}


\NormalTok{ClearCreekDischarge.Monthly_trend <-}\StringTok{ }\KeywordTok{smk.test}\NormalTok{(ClearCreekDischarge.Monthly_ts)}

\CommentTok{# Inspect results}
\NormalTok{ClearCreekDischarge.Monthly_trend}
\end{Highlighting}
\end{Shaded}

\begin{verbatim}
## 
##  Seasonal Mann-Kendall trend test (Hirsch-Slack test)
## 
## data:  ClearCreekDischarge.Monthly_ts
## z = 1.6586, p-value = 0.09719
## alternative hypothesis: true S is not equal to 0
## sample estimates:
##      S   varS 
##    590 126102
\end{verbatim}

\begin{Shaded}
\begin{Highlighting}[]
\KeywordTok{summary}\NormalTok{(ClearCreekDischarge.Monthly_trend)}
\end{Highlighting}
\end{Shaded}

\begin{verbatim}
## 
##  Seasonal Mann-Kendall trend test (Hirsch-Slack test)
## 
## data: ClearCreekDischarge.Monthly_ts
## alternative hypothesis: two.sided
## 
## Statistics for individual seasons
## 
## H0
##                      S  varS    tau      z Pr(>|z|)  
## Season 1:   S = 0   35 10449  0.035  0.333 0.739425  
## Season 2:   S = 0    4 10450  0.004  0.029 0.976588  
## Season 3:   S = 0  204 10450  0.206  1.986 0.047054 *
## Season 4:   S = 0  230 10450  0.232  2.240 0.025081 *
## Season 5:   S = 0  148 10450  0.149  1.438 0.150434  
## Season 6:   S = 0   94 10450  0.095  0.910 0.362951  
## Season 7:   S = 0  -54 10450 -0.055 -0.518 0.604135  
## Season 8:   S = 0  -99 10449 -0.100 -0.959 0.337703  
## Season 9:   S = 0  -90 10450 -0.091 -0.871 0.383958  
## Season 10:   S = 0  64 11154  0.062  0.597 0.550828  
## Season 11:   S = 0  24 10450  0.024  0.225 0.821984  
## Season 12:   S = 0  30 10450  0.030  0.284 0.776650  
## ---
## Signif. codes:  0 '***' 0.001 '**' 0.01 '*' 0.05 '.' 0.1 ' ' 1
\end{verbatim}

\begin{enumerate}
\def\labelenumi{\arabic{enumi}.}
\setcounter{enumi}{12}
\tightlist
\item
  Is there an overall monotonic trend in discharge over time? If so, is
  it positive or negative?
\end{enumerate}

\begin{quote}
We fail to reject the null hypothesis of no overall monotonic trend in
discharge over time with a 5\% level of significance (Hirsch-Slack test,
z = 1.6586, p-value = 0.09719 \textgreater{} 0.05). At a 5\% level of
significance, the sample data does not provide enough evidence to
conclude that there is an overall monotonic trend in discharge over
time.
\end{quote}

\begin{quote}
S is positive = 590, so if the case the null hypothesis was rejected the
trend would have been positive.
\end{quote}

\begin{enumerate}
\def\labelenumi{\arabic{enumi}.}
\setcounter{enumi}{13}
\tightlist
\item
  Are there any monthly monotonic trends in discharge over time? If so,
  during which months do they occur and are they positive or negative?
\end{enumerate}

\begin{quote}
We reject the null hypothesis of no monotonic trend in discharge over
time for the Months of March and April with a 5\% level of significance
(March: Hirsch-Slack test, z = 1.986, Pr(\textgreater\textbar z\textbar)
= 0.047054 \textless{} 0.05. April: Hirsch-Slack test, z = 2.240,
Pr(\textgreater\textbar z\textbar) = 0.025081 \textless{} 0.05). At a
5\% level of significance, the sample data does provide enough evidence
to conclude that there is a positive monotonic trend in discharge over
time in the Months of March and April..
\end{quote}

\hypertarget{reflection}{%
\subsection{Reflection}\label{reflection}}

\begin{enumerate}
\def\labelenumi{\arabic{enumi}.}
\setcounter{enumi}{14}
\tightlist
\item
  What are 2-3 conclusions or summary points about time series you
  learned through your analysis?
\end{enumerate}

\begin{quote}
I learned how time series can be useful for detecting seasonal behaviors
and trends in a response variable that has been tracked over time. Also,
how to use the ts, and stl functions and the importance of studying the
data first for gaps and the frequency the data was recorded. Finally I
learned about options to solve gap issues in the data to be able to
perform a time series analysis.
\end{quote}

\begin{enumerate}
\def\labelenumi{\arabic{enumi}.}
\setcounter{enumi}{15}
\tightlist
\item
  What data, visualizations, and/or models supported your conclusions
  from 15?
\end{enumerate}

\begin{quote}
Basically the whole lab and classes 11 and 12. Every part of them was
super useful.
\end{quote}

\begin{enumerate}
\def\labelenumi{\arabic{enumi}.}
\setcounter{enumi}{16}
\tightlist
\item
  Did hands-on data analysis impact your learning about time series
  relative to a theory-based lesson? If so, how?
\end{enumerate}

\begin{quote}
Yes. Working with the data makes you think about about it and doubts
that you didn't have in class emerge. Trying to solve those doubts by
yourself is a valuable way of learning.
\end{quote}

\begin{enumerate}
\def\labelenumi{\arabic{enumi}.}
\setcounter{enumi}{17}
\tightlist
\item
  How did the real-world data compare with your expectations from
  theory?
\end{enumerate}

\begin{quote}
To my experience with these type of data (in another country), these was
way more complete that the data I am used to. From theory, I believed
the real data was ok except for that gap in the year 2017 of the Eno
River discharge data.
\end{quote}


\end{document}
